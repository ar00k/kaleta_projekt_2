%! Pobieranie PDF
% ctrl + shift + ~
% komenda: upload.sh

%! SKRÓTY KLAWISZOWE
% LINK do skrótów klawiszowych: https://github.com/James-Yu/latex-workshop/wiki/Snippets
% ctrl+alt+j - przeniesienie z kodu do pdf
% ctrl + click - przeniesienie z pdf do kodu (dokument.pdf)
% zaznaczony fragment kodu -> ctrl+l -> ctrl+w
% gdy kuror na sekcji itp. -> cltr + alt + ] - obniżenie sekcji
% gdy kuror na sekcji itp. -> cltr + alt + [ - podniesienie sekcji
% kopia lini kodu -> ctrl + shift + strzałka w dół
\newpage
\section*{Tutorial}		%1

% * SPOSOBY UZYWANIA MAKRA HERE * # 
% ? Listingi
% Parametr #1: Opis listingu (wyświetla sie bezpośrednio pod listingiem)
% Parametr #2 : Nazwa pliku oraz ID do oznaczania (wazne, zeby byl w katalogu kod oraz jego rozszerzenie to txt) 

\ListingFile{OpisPromptu}{prompt}

Tak wstawiamy listingi, za pomocą:

\begin{verbatim}
	ListingFile\{Opis listingu}{nazwa-pliku} 
 \end{verbatim}

wstawiamy listingi z plików z katalogu kod.

Gdzie:
\begin{verbatim}
	{Opis listingu} - opis listingu wyświetlany pod listingiem
	{nazwa-pliku} - nazwa pliku z katalogu kod i jednocześnie ID do oznaczania
\end{verbatim}

Tak oznaczamy listingi \OznaczKod{prompt} w tekście.\\Za pomocą:

\begin{verbatim}
	ListingFile\{Opis listingu}{nazwa-pliku} 
\end{verbatim}

gdzie:

\begin{verbatim}
	{nazwa-pliku} - nazwa pliku z katalogu kod i jednocześnie ID do oznaczania
\end{verbatim}

\clearpage
% ? Zdjęcia
% OPCJONALNY Parametr #1: Szerokość zdjęcia (domyślnie jest to szerokość paragrafu ale jak sie poda w [#1] to wtedy zmienia się na podaną wartość)
% Parametr #2: Nazwa pliku z rozszerzeniem (podajemy katalog w którym jest plik i jego rozszerzenie np. rys/nazwa-rysunku.png)
% Parametr #3: Opis tego co jest na rysunku (wyświetla sie bezpośrednio pod rysunkiem) 
% Parametr #4: Identyfikator rysunku (do oznaczania zdjęć w tekście) 

\fg{rys/nazwa-rysunku.png}{Opis tego co jest na rysunku}{id-rysunku}

jest równe
\fg[\textwidth]{rys/nazwa-rysunku.png}{Opis tego co jest na rysunku}{id-rysunku}
\clearpage
Tak wstawiamy zdjęcia, za pomocą:

\begin{verbatim}
	\fg{szerokość-rysunku}{nazwa-pliku}{Opis rysunku}{id-rysunku} 
 \end{verbatim}

Gdzie:
\begin{verbatim}
	{szerokość-rysunku} - szerokość rysunku domyślnie \textwidth
	{nazwa-pliku} - nazwa pliku z katalogu rys i jego rozszerzenie
	{Opis rysunku} - opis rysunku wyświetlany pod rysunkiem
	{id-rysunku} - identyfikator rysunku
\end{verbatim}

Tak oznaczamy zdjęcia \OznaczZdjecie{id-rysunku} w tekście. Za pomocą:

\begin{verbatim}
	\OznaczZdjecie{id-rysunku}
	\end{verbatim}



\section{Założenia projektowe – wymagania}		%1
\begin{enumerate}
    \item Utworzenie własnej domeny AD według formatu firma.ad, gdzie firma to nazwa firmy dla której
          przygotowujemy projekt.
    \item Autoryzacja pracownika przy użyciu imiennego konta działającego na wszystkich komputerach w
          sieci firmowej. Login zgodny ze schematem: imie.nazwisko. Utworzyć minimum po trzy konta dla
          każdego wydziału. Tworzenie grup oraz kont w domenie wykonywać przez skrypt, który odczyta
          dane grupy, konta z pliku i utworzy w domenie AD.
    \item Pracownicy powinni należeć do grupy globalnej odpowiedniej dla wydziału w którym
          pracują(przewidzieć 5 przykładowych wydziałów, np. kadry, place, gospodarczy, marketing, itp wg.
          własnego uznania). W przedsiębiorstwie przewidziano wydział informatyczny, którego pracownicy
          mają w pełni administrować domeną przedsiębiorstwa. Tworzenie grup oraz kont w domenie
          wykonywać przez skrypt, który odczyta dane grupy, konta z pliku i utworzy w domenie AD.
    \item Pracownicy powinni korzystać z zasobów sieciowych o nazwach: wspolny, oraz zasób wydziałowy
          (oddzielny zasób dla każdego wydziału). Zasoby powinny być udostępnione poprzez klaster pracy
          awaryjnej, który powinien korzystać z przestrzeni dyskowej (macierzy RAID-1) udostępnionej
          poprzez iSCSI o przestrzeni wypadkowej 30GB. Jeśli pozwalają zasoby sprzętowe to programową
          macierz RAID-1, oraz iSCSI Target Serwer można zainstalować na oddzielnym serwerze o nazwie
          SMPXX.firma.ad.
    \item Zasób wspolny ma być mapowany użytkownikowi jako dysk (patrz: założenia projektowej),
          natomiast zasób wydziałowy jako jeden z dysków (patrz: założenia projektowej) adekwatnie do
          grupy wydziałowej w której się pracownik znajduje. Mechanizm mapowania automatyczny przy
          użyciu polis GPO.
    \item System ma umożliwić instalację stacji klienckich z obrazu udostępnionego na serwerze.
    \item Pracownicy powinni mieć dostęp do drukarek sieciowych udostępnianych poprzez serwer wydruków. Serwer wydruków można zainstalować na oddzielnym serwerze, np. o nazwie \texttt{SPRXXfirma.ad}, lub w przypadku małych zasobów, na serwerze \texttt{SDC}. Dostęp do drukarki \texttt{\textbackslash\textbackslash SPRXX-firma.ad\textbackslash nazwa-drukarki}.
          \newpage
    \item Konfiguracja stacji klienckich w sposób automatyczny.
    \item Na klastrze pracy awaryjnej należy wdrożyć DHCP Serwer.
    \item Po udanym wdrożeniu serwera DHCP, wyłączyć serwer DHCP pracujący na SDC
    \item Wdrożyć serwer WWW wraz z firmową stroną (nie domyślną) dostępną pod adresem
          www.firma.ad.
    \item Wdrożyć WordPress współpracujący z usługą IIS Windows Serwera 2022.
\end{enumerate}
\subsection{Założenia projektowe}
\begin{itemize}
    \item nazwy serwerów powinny być zgodne ze schematem, przykładowo: SDCXX-NSACZ.firma.ad. - gdzie
          XX oznacza ostatnie dwie cyfry w numerze albumu,
    \item przyjąć adresację sieci jak niżej: 192.168.XX.40/24 - gdzie XX oznacza ostatnie dwie cyfry w numerze albumu,
    \item Zasoby sieciowe udostępnione na klastrze FC powinny być widoczne dla komputerów w postaci ścieżki UNC jako \texttt{\textbackslash\textbackslash sfsXX.firma.ad\textbackslash zasob}, \\ oraz \texttt{\textbackslash\textbackslash sfsXX.firma.ad\textbackslash wydzial} (nazwa wydziału), gdzie \texttt{XX} oznacza ostatnie dwie cyfry w numerze albumu.
    \item mapowanie dysków zgodnie ze schematem:
          zasób wspolny jako dysk X: , dyski wydziałowe(oddzielny zasób dla każdego wydziału) od G do K,
    \item  zaprojektować polisy tworzącą katalog c:\textbackslash ProjektyXX na stacjach końcowych, oraz ustawiającą
          zmienną środowiskową FIRMA=Nazwa, gdzie XX oznacza ostatnie dwie cyfry w numerze albumu,
    \item zakres adresacji IP dla stacji końcowych w zakresie od 100 ... do 200, uwzględnić funkcjonalność DHCP Serwer’a uruchomionego na klastrze,
    \item na stacji zainstalować drukarkę sieciową udostępnioną przez serwer wydruków i pokazać jej
          działanie.
\end{itemize}
